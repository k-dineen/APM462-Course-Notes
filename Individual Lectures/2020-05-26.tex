\documentclass[11pt]{article}
\usepackage[utf8]{inputenc}
\usepackage{amsmath, amsthm, amssymb, amsfonts, mathtools, tikz-cd, float}
\usepackage[left=2.5cm,right=2.5cm]{geometry}
\usepackage[shortlabels]{enumitem}

\newcommand{\Int}{\text{Int}}
\newcommand{\R}{\mathbb{R}}
\newcommand{\Z}{\mathbb{Z}}
\newcommand{\pd}{\partial}
\renewcommand{\epsilon}{\varepsilon}
\renewcommand{\hat}{\widehat}
\renewcommand{\tilde}{\widetilde}

\newtheorem{theorem}{Theorem}[section]
\newtheorem{corollary}{Corollary}[theorem]
\newtheorem{lemma}[theorem]{Lemma}
\newtheorem{proposition}{Proposition}

\newtheorem{definition}{Definition}

\pagestyle{myheadings}


\begin{document}

\section{Constrained Optimization (May 26)}

Consider the following minimization problem:
\begin{align*}
\text{minimize } &f(x,y) = xy \\
\text{subject to } &x^2 + y^2 \leq 1
\end{align*}
Let $\Omega$ be the feasible set. The feasible directions at a point $(x_0, y_0) \in \Omega$ are the $(v, w) \in \R^2$ such that $(v, w) \cdot (x_0, y_0) < 0$, or $vx_0 + wy_0 < 0$. By the FONC for a minimizer, $\nabla f(x_0, y_0) \cdot (v, w) \geq 0$, so $wx_0 + vy_0 \geq 0$. Note that a local minimum must occur on the boundary. (Why?) We have three cases, depending on the sign of $x_0 + y_0$.
\begin{enumerate}[(i)]
\item $x_0 + y_0 < 0$: can't occur
\item $x_0 + y_0 > 0$: can't occur
\item $x_0 + y_0 = 0$: good!
\end{enumerate}
(This part could not be finished as attention had to be diverted from the lecture.)

\subsection{Second Order Necessary Condition for a Local Minimizer}

\begin{theorem}
(Second order sufficient condition for a local minimizer) Let $f$ be $C^2$ on $\Omega \subseteq \R^n$ and suppose $x_0 \in \Omega$ satisfies
\begin{enumerate}[(i)]
\item $\nabla f(x_0) \cdot v \geq 0$ for all feasible directions $v$ at $x_0$,
\item if $\nabla f(x_0) \cdot v = 0$ for some such $v$, then $v^T \nabla^2 f(x_0) v > 0$.
\end{enumerate}
Then $x_0$ is a local minimizer of $f$ on $\Omega$.
\end{theorem}

\subsection{Optimization with Equality Constraints}

Consider the minimization problem
\begin{align*}
\text{minimize } &f(x,y) \\
\text{subject to } &h(x,y) = x^2 + y^2 - 1 = 0
\end{align*}
Suppose $(x_0, y_0)$ is a local minimizer. Two cases:
\begin{enumerate}
\item $\nabla f(x_0, y_0) \neq 0$: we claim that $\nabla f(x_0, y_0)$ is perpendicular to the tangent space to the unit circle $h^{-1}(\{0\})$ at $(x_0, y_0)$. If this is not the case, then we obtain a contradiction by looking at the level sets of $f$, to which $\nabla f$ is perpendicular. Therefore $\nabla f(x_0, y_0) = \lambda \nabla h(x_0, y_0)$ for some $\lambda$.

\item $\nabla f(x_0, y_0) = 0$: as in the previous case, $\lambda = 0$.
\end{enumerate}
In either case, at a local minimizer, the gradient of the function to be minimized is parallel to the gradient of the constraints.

We now recall some elementary differential geometry.
\begin{definition}
For us, a surface is the set of common zeroes of a finite set of $C^1$ functions. 
\end{definition}
\begin{definition}
For us, a differentiable curve on the surface $M \subseteq \R^n$ is the image of a $C^1$ function $x : (a, b) \to M$.
\end{definition}
\begin{definition}
Let $x(s)$ be a differentiable curve on $M$ that passes through $x_0 \in M$ at time $x(0) = x_0$. The velocity vector $v = \left. \frac{d}{ds} \right|_{s=0} x(s)$ of $x(s)$ at $x_0$ is, for us, said to be a tangent vector to the surface $M$ at $x_0$. The set of all tangent vectors to $M$ at $x_0$ is called the tangent space to $M$ at $x_0$ and is denoted by $T_{x_0}M$.
\end{definition}
\begin{definition}
Let $M = \{x \in \R^n : h_1(x) = \cdots = h_k(x) = 0\}$ be a surface. If $\nabla h_1(x_0), \dots, \nabla h_k(x_0)$ are all linearly independent, then $x_0$ is said to be a regular point of $M$.
\end{definition}
\begin{theorem}
At a regular point $x_0 \in M$, the tangent space $T_{x_0} M$ is given by
\[
T_{x_0} M = \{ y \in \R^n : \nabla \mathbf{h}(x_0)y = 0 \}.
\]
\end{theorem}
\begin{proof}
It's in the book. Use the implicit function theorem.
\end{proof}
\begin{lemma}
Let $f, h_1, \dots, h_k$ be $C^1$ functions on the open set $\Omega \subseteq \R^n$. Let $x_0 \in M = \{ x \in \Omega : h_1(x) = \cdots = h_k(x) = 0 \}$. Suppose $x_0$ is a local minimizer of $f$ subject to the constraints $h_i(x) = 0$. Then $\nabla f(x_0)$ is perpendicular to $T_{x_0}M$.
\end{lemma}
\begin{proof}
Without loss of generality, suppose $\Omega = \R^n$. Let $v \in T_{x_0}M$. Then $v = \left. \frac{d}{ds} \right|_{s=0}x(s)$ for some differentiable curve $x(s)$ in $M$ with $x(0) = x_0$. Since $x_0$ is a local minimizer of $f$, $0$ is a local minimizer of $f \circ x$, so $\nabla f(x_0) \cdot x'(0) = \nabla f(x_0) \cdot v = 0$.
\end{proof}

\end{document}
 