\documentclass[11pt]{article}
\usepackage[utf8]{inputenc}
\usepackage{amsmath, amsthm, amssymb, amsfonts, mathtools, tikz-cd, float, cancel}
\usepackage[left=2.5cm,right=2.5cm]{geometry}
\usepackage[shortlabels]{enumitem}

\newcommand{\Int}{\text{Int}}
\newcommand{\R}{\mathbb{R}}
\newcommand{\Z}{\mathbb{Z}}
\newcommand{\pd}{\partial}
\renewcommand{\epsilon}{\varepsilon}
\renewcommand{\hat}{\widehat}
\renewcommand{\tilde}{\widetilde}

\newtheorem{theorem}{Theorem}[section]
\newtheorem{corollary}{Corollary}[theorem]
\newtheorem{lemma}[theorem]{Lemma}
\newtheorem{proposition}{Proposition}

\newtheorem{definition}{Definition}

\pagestyle{myheadings}


\begin{document}

\section{An Example (August 6)}

\subsection{Recap}

Recall from last lecture the "isoperimetric problem"
\begin{align*}
\text{minimize } F[u] &\coloneqq \int_{-a}^a u(x) \, dx \\
\text{subject to } G[u] &\coloneqq \int_{-a}^a \sqrt{1 + u'(x)^2} \, dx = \ell \\
u \in \mathcal{A} &\coloneqq \{ u : [-a, a] \to \R : u \in C^1([a,b]), u(-a) = u(a) = 0 \}.
\end{align*}
We derived the following Euler-Lagrange equation for a minimizer:
\[
\tag{*}
\lambda^2 \frac{u'(x)^2}{1 + u'(x)^2} = (c_1 - x)^2.
\]
We claim that any solution to (*) must satisfy
\[
(x - c_1)^2 + (u(x) - c_2)^2 = \lambda^2,
\]
which follows from implicit differentiation. Therefore the solution is on a circle of the form $(x - c_1)^2 + (y - c_2)^2 = \lambda^2$. Compare with the isoperimetric inequality, which states that of all the simple closed curves of fixed length, the circle of that circumference encloses the most area.

\subsection{Surface of Revolution Example}

Among all the curves $y = u(x)$ joining $(0, b)$ and $(a, 0)$ which enclose a region of fixed area $S$, find the one which minimizes the surfacea rea of the surface of revolution about the $x$-axis. Our functional is
\[
F[u] \coloneqq \int_0^a  u(x) \sqrt{1 + u'(x)^2} \, dx,
\]
which we wish to minimize under the constraint
\[
G[u] \coloneqq \int_0^a u(x) \, dx = S,
\]
where $u \in \mathcal{A} = \{ u \in C^1([0, a]) : u(0) = b, u(a) = 0 \}$. (We cancel the $2\pi$ from the definition of $F$ for simplicity.) The Lagrangians are $L^F(x,z,p) = z\sqrt{1 + p^2}$ and $L^G(x,z,p) = z$. The Euler-Lagrange equation is
\[
-\frac{d}{dx} (L^F_p + \lambda L^G_p) + (L^F_z + \lambda L^G_z) \equiv 0,
\]
with the linear combinations of the Lagrangians evaluated at $(x, u(x), u'(x))$. We obtain
\[
-\frac{d}{dx} \left( \frac{u(x)u'(x)}{\sqrt{1 + u'(x)^2}} + \lambda 0 \right) + \sqrt{1 + u'(x)^2} + \lambda = 0.
\]
A computation shows that a function of the form $u(x) = \alpha x + \beta$ solves these Euler-Lagrange equations. Plugging in the boundary conditions and solving for $\alpha, \beta$ in terms of $a, b$ gives
\[
\frac{x}{a} + \frac{y}{b} = 1.
\]

\textbf{These notes are somewhat incomplete.}


\end{document}
 